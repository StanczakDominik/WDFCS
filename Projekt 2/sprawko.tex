\documentclass[11pt]{article}
\usepackage{amsmath} % AMS Math Package
\usepackage{polski}
\usepackage[utf8]{inputenc}
\usepackage{listings}
\usepackage{graphicx} % Allows for eps images
\newcommand{\png}[1]{\begin{center}\includegraphics{#1}\end{center}}
\newcommand{\largepng}[1]{\begin{center}\includegraphics[width=\linewidth]{#1}\end{center}}

\begin{document}
\largepng{tabelka.png}
\section{Zadanie 1}

Dla przejścia elektronu z poziomu
\[E_{foton} = E_g + E_1n + E_1p\]

Określenie grubości studni
Określenie przerw energetycznych materiałów
Wybór materiałów ze wskazaniem ich stałych sieciowych
Rysunek - schemat struktury
Wskazanie koloru emitowanego światła
\section{Zadanie 2}

Dla danych z zestawu 3 współczynnik kształtu $l/S$ wynosi $1.04mm/1.89mm^2 = 0.05502/mm$.
Dane z pliku można przedstawić na wykresie Arrheniusa tak jak poniżej:
\largepng{z2.png}

Obliczenie współczynników przeprowadza się przy pomocy następującego skryptu w Pythonie:
\begin{lstlisting}[language=Python]
k = 1.38064852e-23
e = 1.60217662e-19
data = numpy.loadtxt("zad2_is_3_readable.txt")
Kelvin = 273.15
L=grubosc = 1.04e-3
S=powierzchnia = 18.9e-6
print(L/S)

T, R = temperatureC, resistanceOhm = data[:,0], data[:,1]
T+=Kelvin
G=1/R
conductivity = G*L/S

def linear_curve(x, a, b):
    return a*x+b
x = 1/T
y = np.log(conductivity*T/S)
coefficients, covs = scipy.optimize.curve_fit(linear_curve, x, y)
print(coefficients)
print(coefficients[0]*k/e)
\end{lstlisting}

Parametry dopasowania prostej $y=ax+b$ do zbioru danych $(1/T, \log{(\sigma T/S)}$ wynoszą:
\[a = 7153.0919 K, b = 26.6597\]

Co pozwala nam, mnożąc przez stałą Boltzmanna oraz dzieląc przez ładunek elektronu, otrzymać
\[ E_a = 0.6164 eV \]

Podstawiając otrzymane wartości ($\sigma_0 = \exp{b}$), otrzymujemy

\[ \sigma(25^{\circ}C) = 0.04834 S/m \]
% Wyznaczenie współczynnika kształtu próbki (tj. l/S)
% Wykreślenie danych w żądanych współrzędnych
% Oznaczenie wykresu (jednostki i opisy osi)
% Dopasowanie prostej i wyznaczenie parametrów dopasowania
Obliczenie energii aktywacji
Obliczenie wartości przewodności w 25 stopni C
\section{Zadanie 3}
Przeksztalcenie i otrzymanie układu równań
Wyznaczenie wyznacznika
Rozwiązanie równania dwukwadratowego
Estetyczne wykreślenie otrzymanych funkcji dla pierwszej strefy Brillouina
\largepng{z3.png}
\end{document}
